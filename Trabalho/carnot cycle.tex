\documentclass[12pt, twocolumn]{article}
\usepackage[portuguese]{babel}
\usepackage[T1]{fontenc}
\usepackage{amsmath, amsfonts, amssymb}
\usepackage{graphicx}
\usepackage{hyperref}
\usepackage[numbers]{natbib}
\usepackage{setspace}
\usepackage{geometry}
\usepackage{parskip}
\usepackage{titlesec}
\usepackage{indentfirst}
\usepackage{authblk}
\usepackage{enumitem}
\usepackage{newtxtext, newtxmath}
\usepackage{cleveref}
\usepackage{abstract}
\usepackage{hyphenat}

\crefname{equation}{Eq.}{Eq.}

\geometry{a4paper, left=1.5cm, right=1.5cm, top=2cm, bottom=2cm, columnsep=1cm}
\setstretch{1.1}

\titleformat{\section}
  {\normalfont\large\bfseries}{\thesection.}{0.5em}{}
\titleformat{\subsection}
  {\normalfont\normalsize\bfseries}{\thesubsection.}{0.5em}{}

\renewcommand{\abstractname}{}
\setlength{\absleftindent}{0pt}
\setlength{\absrightindent}{0pt}

\setlist[enumerate,1]{label=\Alph*:, ref=\Alph*}

\setlength{\parindent}{1cm}

\renewcommand\Authands{ e }
\renewcommand\Authand{ e }

\title{Motor de Carnot e os Limites Termodinâmicos da Eficiência}
\author{Alice Reis}
\author{João Guilherme}
\author{Lucas Jalles}
\author{Luíza Lee}
\author{Marcela Messala}
\author{Sophia Mayumi}
\affil{CAP-UFRJ}
\date{Outubro 2025}

\begin{document}

\twocolumn[
\begin{@twocolumnfalse}
\maketitle
\begin{abstract}
\noindent

 No presente arquivo, buscamos apresentar e explicar o funcionamento do motor de Carnot. Tal ciclo termodinâmico 
 foi teorizado por Sadi Carnot com o intuito de estabelecer a máxima eficiência dos motores térmicos operando
 entre dois reservatórios de temperatura, um de alta temperatura ($T_a$), e outro de baixa ($T_b$). Discutiremos 
 as quatro etapas e derivaremos a equação fundamental da eficiência $\eta = 1 - T_b/T_a$ para então explicar a 
 impossibilidade teórica de realmente alcançar 100\% de eficiência. Além disso, analisaremos as contradições às 
 leis da termodinâmicas do motor.

\end{abstract}
\bigskip
\end{@twocolumnfalse}
]

\section{Introdução}

Sadi Carnot (1796-1832) foi um importante cientista francês, filho de um dos generais de Napoleão Bonaparte. 
Em sua obra de 1824, Réflexions sur la puissance motrice du feu et sur les machines propres à développer cette 
puissance (Reflexões sobre a Potência Motriz do Fogo e Máquinas Próprias para Aumentar essa Potência), o 
físico discute a importância dos motores a vapor e apresenta seu motor reversível, cuja eficiência dependia 
apenas das temperaturas de fontes quentes e frias.

O modelo de Carnot foi importante, e é até hoje, pois foi ele o utilizado para fazer as máquinas do século XIX 
na Revolução Industrial. Além disso, até hoje ele é o modelo mais usado para fabricação de motores no geral.

A partir disso, foi possível mostrar que era impossível criar uma máquina com 100\% de eficiência, 
devido principalmente a não se adequar à primeira lei da termodinâmica.

\section{O Motor de Carnot}

Carnot foi pioneiro na pesquisa de um modelo de motor 100\% eficiente. Ele seria constituído por um cilindro 
cujo único orifício separando o meio interno do externo ficaria na parte inferior. Dentro, haveria um fluído
ideal (fluido ideal ou perfeito; de viscosidade nula) e um pistão que comprimiria o fluido, exercendo nele 
pressão. O funcionamento seria baseado em um estômago de unicórnio, onde o mundo é perfeito: Não seria possível 
entrar ou sair calor em qualquer outro lugar dessa máquina, desprezando suas trocas de calor 
(seja por condução, convecção ou irradiação) e o atrito do sistema.

No modelo, existiria uma força externa empurrando o pistão, uma barra de alta temperatura e outra de baixa 
temperatura. Ambas as temperaturas não deveriam variar ao longo do experimento, mesmo sob calor. Tendo que ser 
suficientemente grandes para tal.

\subsection{As quatro etapas do Ciclo de Carnot}

No primeiro momento, se colocaria a barra quente no orifício que permite trocas de calor, ao mesmo tempo que se 
retiraria alguma pressão sob o pistão. O gás se expandiria, mas não esfriaria, visto que a barra quente 
transferiria $Q_1$ para o gás, mantendo sua temperatura. O processo seguirá o de uma expansão isotérmica, ou seja,
sem variação na temperatura.

Em seguida, se tiraria mais pressão e, também, a barra quente. O gás continuaria a se expandir, mas agora, 
sem a barra quente, esfriaria. Com a variação, então, de temperatura, mas não de calor, o processo é nomeado de 
expansão adiabática.

Quando atingisse a mesma temperatura da barra fria, essa seria colocada na entrada de calor, e a força externa 
no pistão seria aumentada, comprimindo o gás. Ao ser comprimido, se esperaria aumento na temperatura, mas tal 
não se segue, pois o calor $Q_2$ vai para a barra fria. Novamente, sem variação de temperatura, o terceiro 
processo é chamado de compressão isotérmica.

O quarto e último processo é a compressão adiabática. Similarmente ao segundo processo, a fonte de calor 
(que agora age como um ralo de calor) é retirada, mas inversamente a ele, adicionamos pressão para que 
haja, na verdade, compressão (note que a pressão total após essa última adição será igual à inicial). 
Sem o ralo de calor, o gás, comprimindo, esquentará até a temperatura inicial.

No fim de todos os estágios, o sistema e todos os seus parâmetros voltam ao estado que se encontravam 
anteriormente ao processo um. Dizemos então que o experimento é um ciclo com todos os seus estágios sendo 
reversíveis. O Ciclo de Carnot é constituído por essas duas expansões iniciais e então duas compressões, 
alternando entre processos isotérmicos e adiabáticos. Durante os processos isotérmicos, há transferência 
de calor $Q_1$ e $Q_2$ para dentro e fora do fluido respectivamente.

A diferença entre os valores $Q_1$ e $Q_2$ é o coração da eficiência do motor. 
Usando da proporcionalidade, podemos dizer que a diferença entre a temperatura máxima e mínima é similarmente 
conectada à eficiência. E essa relação pode ser descrita por uma equação que nos ajudará a entender o porquê 
do motor ser impossível de ser recriado.

\subsection{Derivação da equação de eficiência}

Para chegar no cálculo da eficiência, precisamos antes entender o que essa significa e como calculá-la. 
Para isso, pense em lucro, A diferença entre a arrecadação e o prejuízo. Similarmente, a eficiência 
representará quanto de energia útil (arrecadação), nesse caso, trabalho, o motor produzirá ao consumir 
energia que damos ao sistema, nesse caso, $Q_1$ (prejuízo). 

Calculamos então a razão entre os dois para obter o resultado como uma porcentagem. Claro que, diferentemente 
do caso do lucro, não poderemos arrecadar mais energia do que investimos. Não se é possível criar energia. 
Por tanto, podemos apenas tentar se aproximar o máximo possível do 100\%. Sendo essa a premissa do motor 
de Carnot.

Tendo isso em mente, descreveremos a eficiência como:

\begin{equation}\label{eq:eficiencia}
\text{Eficiência} = \frac{W_{\text{total}}}{Q_1}
\end{equation}

Calcularemos então, o trabalho feito em cada estágio. Para os processos isotérmicos podemos estabelecer:

\begin{equation}\label{eq:work_iso_general}
W_{\text{isot.}} = \int_{V_i}^{V_f} P \,dV
\end{equation}

E considerando a lei dos gases ideias:

\begin{equation}\label{eq:ideal_gas_law}
PV=nRT, \quad P=\frac{nRT}{V}
\end{equation}

Substituímos:

\begin{equation}\label{eq:work_iso_subst}
W_{\text{isot.}} = \int_{V_i}^{V_f} \frac{nRT}{V} \,dV
\end{equation}

Tirando as constantes da integração, resta apenas integrar dV/V:

\begin{equation}\label{eq:work_iso_integrated}
W_{\text{isot.}} = nRT\ln\left({\frac{V_f}{V_i}}\right)
\end{equation}

É possível concluir então, que para o processo de expansão isotérmica teremos:

\begin{equation}\label{eq:work_1to2}
W_{1\rightarrow2} = nRT_{a}\ln\left({\frac{V_2}{V_1}}\right) =Q_{1\rightarrow2}
\end{equation}

Onde, da primeira lei da termodinâmica, podemos dizer que $W = Q$, já que não há diferença na temperatura (e por tanto, $\Delta U = 0$). Para o processo de compressão isotérmica 
obteremos, similarmente:

\begin{equation}\label{eq:work_3to4}
W_{3\rightarrow4} = nRT_{b}\ln\left({\frac{V_4}{V_3}}\right)=Q_{3\rightarrow4}
\end{equation}

Partiremos então para um melhor entendimento da relação entre calor e trabalho para os processos adiabáticos. 
Para isso, recuperaremos a definição de capacidade térmica molar à volume constante:

\begin{equation}\label{eq:cv_definition}
C_v = \frac{1}{n}\,\left(\frac{\partial Q}{\partial T}\right)_V
\end{equation}

Se o volume é constante, podemos inferir que $\ W = 0$ e, por tanto, podemos igualar dU e dQ. 
Isolando dU, integramos de ambos os lados para obter:

\begin{equation}\label{eq:internal_energy}
dU_v = dQ_v = nC_vdT,
\end{equation}

\begin{equation}\label{eq:internal_energy_2}
\int_{U_i}^{U_f}dU = \int_{Ti}^{Tf}nC_v\,dT \rightarrow \Delta U = nC_v\Delta T
\end{equation}

Embora pareça contraintuitivo, podemos usar essa relação mesmo para gases em expansão e compressão, 
contanto que sejam ideais. Isso se dá pois, para esses gases, a energia interna é uma função da temperatura apenas. 
Sendo possível desprezar a diferença de volume.

Considerando $Q = 0$ para os processos adiabáticos, teremos que $|W| = |\Delta U|$ . 
E como acabamos de demonstrar esse valor, basta substituir em:

\vspace{2cm}

\begin{equation}\label{eq:work_2to3}
W_{2\rightarrow3} = -nC_v(T_b-T_a) = -\Delta U_{2\rightarrow3},
\end{equation}

\begin{equation}\label{eq:work_4to1}
W_{4\rightarrow1} = -nC_v(T_a-T_b) = -\Delta U_{4\rightarrow1}
\end{equation}

Note que na \cref{eq:work_2to3} o trabalho está sendo feito pelo gás, representando direta e exclusivamente a 
diminuição de sua energia interna. Mas então, W não deveriam ser positivo? Ele é! 
Perceba que $(T_b-T_a)<0$ e, por tanto, o termo inteiro é positivo!

Note também que o trabalho e variação de energia interna para os dois processos adiabáticos são um o inverso 
do outro. Essa relação será importante na hora de calcular a eficiência, já que, por isso, não precisaremos representá-los.

Agora que definimos os parâmetros mais importantes das quatro etapas, voltaremos à \cref{eq:eficiencia}, e 
substituiremos o $W_{\text{total}}$ pela soma do trabalho feito pelo gás ao longo dos processos isotérmicos 
(lembrando que os processos adiabáticos se anulam), e representaremos $Q_1$ por seu valor dado na \cref{eq:work_1to2}:

\begin{equation}\label{eq:eficiencia_expandida}
\text{Eficiência} = \frac{nRT_{a}\ln\left({\frac{V_2}{V_1}}\right) + nRT_{b}\ln\left({\frac{V_4}{V_3}}\right)}{nRT_{a}\ln\left({\frac{V_2}{V_1}}\right)}
\end{equation}

Considerando que o processo geratriz dos termos foi de expansão adiabática, podemos escrever:

\begin{equation}\label{eq:adiabatic1}
T_3 V_3^{\gamma-1} = T_4 V_4^{\gamma-1},\; \text{onde}
\end{equation}

\begin{equation}\label{eq:gamma_def}
\gamma = \frac{C_p}{C_v}
\end{equation}

A fim de simplificar Cp, consideremos uma transformação à pressão constante. Então, 
Da primeira lei da termodinâmica, e da definição de trabalho, podemos escrever:

\begin{equation}\label{eq:qp_geral}
Q_p=\Delta U_p +W_p=\Delta U_p+P\Delta V
\end{equation}

Onde conhecemos sobre as relações dos gases ideais, e por motivos já discutidos, que:

\begin{equation}\label{eq:ideal_gas_again}
PV=nRT,
\end{equation}

\begin{equation}\label{eq:dU_relation}
\Delta U_p = \Delta U_v = nC_v\Delta T    
\end{equation}

Substituindo as equações \cref{eq:dU_relation} e \cref{eq:ideal_gas_again} na \cref{eq:qp_geral}, obtemos:

\begin{equation}\label{eq:qp_calculated}
Q_p = nC_v\Delta T + nR\Delta T = n(C_v+R)\Delta T
\end{equation}

E sabemos que $Q_p = nC_p\Delta T$ (por derivação similar à que fizemos para $C_v$), por tanto, 
podemos afirmar que $C_p = C_v + R$\,.

Finalmente, com esse resultado, podemos escrever, da \cref{eq:adiabatic1} e \cref{eq:gamma_def}:

\begin{equation}\label{eq:adiabatic_rel1}
\left(\frac{T_2}{T_3}\right)^{C_v/R} = \frac{V_3}{V_2}
\end{equation}

\begin{equation}\label{eq:adiabatic_rel2}
\left(\frac{T_1}{T_4}\right)^{C_v/R} = \frac{V_4}{V_1}
\end{equation}

A partir da natureza do Ciclo, adicionamos que $T_1 = T_2$ e $T_3 = T_4$. Igualando então as equações e 
substituindo os termos, concluímos que:

\begin{equation}\label{eq:volume_relation}
\frac{V_3}{V_4} = \frac{V_2}{V_1}
\end{equation}

Substituindo a \cref{eq:volume_relation} na \cref{eq:eficiencia_expandida} obtemos:

\begin{equation}\label{eq:eficiencia_simplificada_2}
\text{Eficiência} = \frac{nRT_{a}\ln\left({\frac{V_2}{V_1}}\right) + nRT_{b}\ln\left({\frac{V_2}{V_1}}\right)}{nRT_{a}\ln\left({\frac{V_2}{V_1}}\right)}
\end{equation}

Cancelando os termos semelhantes, chegamos a:

\begin{equation}\label{eq:eficiencia_simplificada}
\text{Eficiência} = \frac{T_a - T_b}{T_a}
\end{equation}

Ou ainda:

\begin{equation}\label{eq:eficiencia_final}
\eta = 1 - \frac{T_b}{T_a}
\end{equation}

A partir da \cref{eq:eficiencia_final}, observamos então que a eficiência só será 1 (100\%), 
se ${T_b}/{T_a} = 0$. Ou seja, se $T_b = 0$ ou $T_a = \infty$. Ou seja, se a temperatura do reservatório mais frio 
for zero absoluto ou a temperatura do reservatório mais quente for infinita. 

\section{Porque um Motor não pode ter 100\% de eficiência}

Um motor nunca pode chegar a 100\% de eficiência pois para isso seria necessário contraditar 2 leis da termodinâmica.

\begin{enumerate}
\item O universo tende ao caos e à entropia, ou seja, a energia sempre quer se dispersar. E quando ela se 
dispersa não pode mais ser usada para gerar trabalho.
\item Energia tem que vir de algum lugar e deve ir a algum lugar, ou seja, não é possível criar energia do 
nada ou mantê-la no mesmo sistema para sempre.
\end{enumerate}

Para que um motor tivesse 100\% de eficiência, seria necessário que a fonte de calor tivesse um valor 
infinitamente alto ou que a fonte fria tivesse o valor igual a 0 
(e um lugar infinitamente grande para o gás expandir). Isso pode ser observado na \cref{eq:eficiencia_final}.

Se $T_b=0$, então $0/\text{qualquer coisa}=0$ e $1-0=100\%$. Se $T_a=\infty$, qualquer coisa/$\infty$ tende a 0 e
, portanto, Eficiência$=100\%$.

Não existe uma fonte de calor infinita, pois seria necessário algo que vai contra a lei de que a energia tem 
que vir de algum lugar, já que não há energia infinita no universo.

E não existe uma forma de atingir 0 graus Kelvin até onde nós sabemos. Nós conseguimos chegar bem próximo 
em um sistema fechado, mas quanto mais próximo mais difícil. 
Também é impossível criar um recipiente infinitamente longo para que o gás se expanda até 0 Kelvin.

Terceiramente: As condições para chegar a 100\% de eficiência também ignoram muitas leis do nosso mundo. 
Como por exemplo a troca de calor entre o gás e o recipiente, o atrito entre as moléculas ou irradiação 
através do recipiente, e o uso de um gás ideal. Portanto, um motor 100\% eficiente no mundo em que 
vivemos não passa de um sonho.

\section{Como seria um motor 100\% eficiente?}

Um motor 100\% eficiente precisaria de um sistema que não interage com o mundo fora dele em nenhuma forma.

Além disso, precisariamos escolher um entre dois caminhos:

\begin{enumerate}
\item Fonte de calor infinita: Se tivermos algo que gera calor infinito, a quantidade de calor que 
escapa do sistema seria irrelevante pois ele é infinito e sua eficiência seria de 100\%.
\item Fonte de frio capaz de fazer o sistema chegar a 0 Kelvin 
(precisamos também de um recipiente infinitamente grande): 
Se formos capazes de chegar a 0 Kelvin, não perderíamos energia para realizar nenhuma parte do ciclo, 
pois a pressão comprimiria o gás sozinha. Portanto, Eficiência $=100\%$.
\end{enumerate}

Em uma realidade distante da nossa, onde um desses caminhos é possível, 
viável, e verdadeiramente proveitoso, o sistema solucionaria muitos 
problemas enormes de nossas vidas. Vivemos em uma era 
onde energia elétrica é uma necessidade básica e é 
demandada cada vez mais. Um motor 100\% eficiente nos permitiria gerar 
muito mais energia pra suprir nossa demanda, aumentar a acessibilidade 
e reduzir o custo da mesma.

Não só isso, eles diminuiriam o aquecimento global em muito já que 
queimaríamos menos combustíveis fosseis e carros elétricos seriam mais 
viáveis e baratos. 

Eles também não superaqueceriam, pois não há calor vazando, mantendo a 
temperatura do ciclo constante.

\section{Conclusões}

O ciclo de Carnot estabelece a eficiência máxima teórica para qualquer motor térmico operando 
entre dois reservatórios de temperatura. Nossa análise demonstra que alcançar 100\% de eficiência requer 
condições fisicamente impossíveis: ou uma fonte de calor com temperatura infinita ou um sumidouro frio a 
zero absoluto. Estes requisitos violam princípios termodinâmicos fundamentais, particularmente a primeira e 
segunda leis da termodinâmica. Embora o motor de Carnot permaneça uma construção teórica inestimável para entender os limites 
da eficiência termodinâmica, motores práticos devem sempre operar abaixo deste máximo ideal devido à inevitável 
dissipação de energia e aumento de entropia.

\begin{thebibliography}{9}
\bibitem{carnot1824} Carnot, S. (1824). \emph{Réflexions sur la puissance motrice du feu et sur les machines propres à développer cette puissance}. Bachelier.
\bibitem{callen1985} Callen, H. B. (1985). \emph{Thermodynamics and an Introduction to Thermostatistics}. John Wiley \& Sons.
\bibitem{atkins2010} Atkins, P., \& de Paula, J. (2010). \emph{Physical Chemistry}. Oxford University Press.
\bibitem{santos1999} Santos, F. C. (1999). \emph{Termodinâmica}. Editora da Universidade de São Paulo.
\end{thebibliography}

\end{document}