\documentclass[10pt, twocolumn]{article}
\usepackage[portuguese]{babel}
\usepackage[T1]{fontenc}
\usepackage{amsmath, amsfonts, amssymb}
\usepackage{graphicx}
\usepackage{hyperref}
\usepackage[numbers]{natbib}
\usepackage{setspace}
\usepackage{geometry}
\usepackage{titlesec}
\usepackage{indentfirst}
\usepackage{authblk}
\usepackage{enumitem}
\usepackage{lmodern}
\usepackage{cleveref}
\usepackage{abstract}
\usepackage{hyphenat}
\usepackage{float}
\usepackage{microtype}

\crefname{equation}{equação}{equação}

\hypersetup{
  colorlinks=true,
  linkcolor=magenta,
  citecolor=red,
  urlcolor=cyan
  }
  
  \geometry{a4paper, left=1.5cm, right=1.5cm, top=2cm, bottom=2cm, columnsep=1cm}
  \setstretch{1.1}
  
  \renewcommand{\thesection}{\Roman{section}.}
  \renewcommand{\thesubsection}{\Alph{subsection}.}
  
  \titleformat{\section}
  {\centering\normalfont\normalsize\bfseries}{\thesection}{1em}{\MakeUppercase}
  \titlespacing*{\section}{0pt}{1.2ex}{0.6ex}
  
  \titleformat{\subsection}
  {\centering\normalfont\small\bfseries}{\thesubsection}{1em}{\MakeUppercase}
  \titlespacing*{\section}{0pt}{1.2ex}{0.6ex}
  
  \renewcommand{\abstractname}{}
  \setlength{\absleftindent}{0pt}
  \setlength{\absrightindent}{0pt}
  
  \setlist[enumerate,1]{label=\Alph*:, ref=\Alph*}
  
  \setlength{\parindent}{1cm}
  
  \renewcommand\Authands{ e }
  
\title{Motor de Carnot e os Limites Termodinâmicos da Eficiência}
\author{Alice Reis}
\author{João Guilherme}
\author{Lucas Jalles}
\author{Luíza Lee}
\author{Marcela Messala}
\author{Sophia Mayumi}
\affil{CAP-UFRJ}
\date{Outubro 2025}

\tolerance 1414
\hbadness 1414
\emergencystretch 1.5em

\hyphenpenalty=500
\exhyphenpenalty=10000

\setlength{\parindent}{1cm}
\setlength{\parskip}{3pt}

% \raggedbottom

\begin{document}

\twocolumn[
  \begin{@twocolumnfalse}
    \maketitle
    \begin{abstract}
    \noindent
    No presente arquivo, buscamos apresentar e explicar o funcionamento do motor de Carnot. Tal ciclo termodinâmico 
    foi teorizado por Sadi Carnot com o intuito de estabelecer a máxima eficiência dos motores térmicos operando
    entre dois reservatórios de temperatura, um de alta temperatura ($T_a$), e outro de baixa ($T_b$). Discutiremos
    as quatro etapas e derivaremos a equação fundamental da eficiência $\eta = 1 - T_b/T_a$ para então explicar a        
    impossibilidade teórica de realmente alcançar 100\% de eficiência. Além disso, analisaremos as contradições às 
    leis da termodinâmicas do motor.
    \end{abstract}
    \bigskip
  \end{@twocolumnfalse}
  ]
  
  \section{Introdução}
  
  Sadi Carnot (1796-1832) foi um importante cientista francês, filho de um dos generais de Napoleão Bonaparte. 
  Em sua obra de 1824, Réflexions sur la puissance motrice du feu et sur les machines propres à développer cette 
  puissance (Reflexões sobre a Potência Motriz do Fogo e Máquinas Próprias para Aumentar essa Potência), o 
  físico discute a importância dos motores a vapor e apresenta seu motor reversível, cuja eficiência dependia 
  apenas das temperaturas de fontes quentes e frias.

  O modelo de Carnot foi extremamente importante para a construçoes de maquinas na revolução 
  industrial, e até hoje ele é um modelo de conceito básico para a construção de diversas
  máquinas. A partir dele, foi possível mostrar que era impossível criar uma máquina 
  com 100\% de eficiência, por não se adequar a primeira lei da termodinâmica e não 
  seguir o funcionamento básico das trocas de calor.

  Explicaremos mais a fundo o porque da impossibilidade, a partir
  da equação fundamental da eficiência. Essa, que derivaremos de duas
  formas distintas, dentre elas, a partir do diagrama P-V, que plotamos
  usando Python. 


\section{O Motor de Carnot}

Carnot foi pioneiro na pesquisa de um modelo de motor 100\% eficiente. Ele seria constituído por um cilindro 
cujo único orifício separando o meio interno do externo ficaria na parte inferior. Dentro, haveria um fluído
ideal (fluido ideal ou perfeito; de viscosidade nula) e um pistão que comprimiria o fluido, exercendo nele 
pressão. O funcionamento seria baseado em um estômago de unicórnio, onde o mundo é perfeito: Não seria possível 
entrar ou sair calor em qualquer outro lugar dessa máquina, desprezando suas trocas de calor 
(seja por condução, convecção ou irradiação) e o atrito do sistema.

No modelo, existiria uma força externa empurrando o pistão, uma barra de alta temperatura e outra de baixa 
temperatura. Ambas as temperaturas não deveriam variar ao longo do experimento, mesmo sob calor. Tendo que ser 
suficientemente grandes para tal.

\subsection{As quatro etapas do Ciclo de Carnot}

No primeiro momento, se colocaria a barra quente no orifício que permite trocas de calor, ao mesmo tempo que se 
retiraria alguma pressão sob o pistão. O gás se expandiria, mas não esfriaria, visto que a barra quente 
transferiria $Q_1$ para o gás, mantendo sua temperatura. O processo seguirá o de uma expansão isotérmica, ou seja,
sem variação na temperatura.

Em seguida, se tiraria mais pressão e, também, a barra quente. O gás continuaria a se expandir, mas agora, 
sem a barra quente, esfriaria. Com a variação, então, de temperatura, mas não de calor, o processo é nomeado de 
expansão adiabática.

Quando atingisse a mesma temperatura da barra fria, essa seria colocada na entrada de calor, e a força externa 
no pistão seria aumentada, comprimindo o gás. Ao ser comprimido, se esperaria aumento na temperatura, mas tal 
não se segue, pois o calor $Q_2$ vai para a barra fria. Novamente, sem variação de temperatura, o terceiro 
processo é chamado de compressão isotérmica.

O quarto e último processo é a compressão adiabática. Similarmente ao segundo processo, a fonte de calor 
(que agora age como um ralo de calor) é retirada, mas inversamente a ele, adicionamos pressão para que 
haja, na verdade, compressão (note que a pressão total após essa última adição será igual à inicial). 
Sem o ralo de calor, o gás, comprimindo, esquentará até a temperatura inicial.

No fim de todos os estágios, o sistema e todos os seus parâmetros voltam ao estado que se encontravam 
anteriormente ao processo um. Dizemos então que o experimento é um ciclo com todos os seus estágios sendo 
reversíveis. O Ciclo de Carnot é constituído por essas duas expansões iniciais e então duas compressões, 
alternando entre processos isotérmicos e adiabáticos. Durante os processos isotérmicos, há transferência 
de calor $Q_1$ e $Q_2$ para dentro e fora do fluido respectivamente.

A diferença entre os valores $Q_1$ e $Q_2$ é o coração da eficiência do motor. 
Usando da proporcionalidade, podemos dizer que a diferença entre a temperatura máxima e mínima é similarmente 
conectada à eficiência. E essa relação pode ser descrita por uma equação que nos ajudará a entender o porquê 
do motor ser impossível de ser recriado.

\subsection{Derivação da equação de eficiência}

Para chegar no cálculo da eficiência, precisamos antes entender o que essa significa e como calculá-la. 
Para isso, pense em lucro, A diferença entre a arrecadação e o prejuízo. Similarmente, a eficiência 
representará quanto de energia útil (arrecadação), nesse caso, trabalho, o motor produzirá ao consumir 
energia que damos ao sistema, nesse caso, $Q_1$ (prejuízo). 

Calculamos então a razão entre os dois para obter o resultado como uma porcentagem. Claro que, diferentemente 
do caso do lucro, não poderemos arrecadar mais energia do que investimos. Não se é possível criar energia. 
Portanto, podemos apenas tentar se aproximar o máximo possível do 100\%. Sendo essa a premissa do motor 
de Carnot.

Tendo isso em mente, descreveremos a eficiência como:

\begin{equation}\label{eficiencia}
\text{Eficiência} = \frac{W_{\text{total}}}{Q_1}
\end{equation}

Calcularemos então, o trabalho feito em cada estágio. Para os processos isotérmicos podemos estabelecer:

\begin{equation}\label{trab_iso}
W_{\text{isot.}} = \int_{V_i}^{V_f} P \,dV
\end{equation}

E considerando a lei dos gases ideias:

\begin{equation}\label{lei_gases_ideais}
PV=nRT, \quad P=\frac{nRT}{V}
\end{equation}

Substituímos:

\begin{equation}\label{trab_iso_subst}
W_{\text{isot.}} = \int_{V_i}^{V_f} \frac{nRT}{V} \,dV
\end{equation}

Tirando as constantes da integração, resta apenas integrar dV/V:

\begin{equation}\label{trab_iso_integrado}
W_{\text{isot.}} = nRT\ln\left({\frac{V_f}{V_i}}\right)
\end{equation}

É possível concluir então, que para o processo de expansão isotérmica teremos:

\begin{equation}\label{trab_1pra2}
W_{1\rightarrow2} = nRT_{a}\ln\left({\frac{V_2}{V_1}}\right) =Q_{1\rightarrow2}
\end{equation}

Onde, da primeira lei da termodinâmica, podemos dizer que $W = Q$, já que não há diferença na temperatura (e portanto, $\Delta U = 0$). Para o processo de compressão isotérmica 
obteremos, similarmente:

\begin{equation}\label{trab_3pra4}
W_{3\rightarrow4} = nRT_{b}\ln\left({\frac{V_4}{V_3}}\right)=Q_{3\rightarrow4}
\end{equation}

Partiremos então para um melhor entendimento da relação entre calor e trabalho para os processos adiabáticos. 
Para isso, recuperaremos a definição de capacidade térmica molar à volume constante:

\begin{equation}\label{def_cv}
C_v = \frac{1}{n}\,\left(\frac{\partial Q}{\partial T}\right)_V
\end{equation}

Se o volume é constante, podemos inferir que $\ W = 0$ e, portanto, podemos igualar dU e dQ. 
Isolando dU, integramos de ambos os lados para obter:

\begin{equation}\label{energia_interna}
dU = dQ_v = nC_vdT,
\end{equation}

\begin{equation}\label{energia_interna_2}
\int_{U_i}^{U_f}dU = \int_{Ti}^{Tf}nC_v\,dT \rightarrow \Delta U = nC_v\Delta T
\end{equation}

Embora pareça contraintuitivo, podemos usar essa relação mesmo para gases em expansão e compressão, 
contanto que sejam ideais. Isso se dá pois, para esses gases, a energia interna é uma função da temperatura apenas. 
Sendo possível desprezar a diferença de volume.

Considerando $Q = 0$ para os processos adiabáticos, teremos que $|W| = |\Delta U|$ . 
E como acabamos de demonstrar esse valor, basta substituir em:

\vspace{-0.5\baselineskip}

\begin{equation}\label{trab_2pra3}
W_{2\rightarrow3} = -nC_v(T_b-T_a) = -\Delta U_{2\rightarrow3},
\end{equation}

\vspace{-0.5\baselineskip}

\begin{equation}\label{trab_4pra1}
W_{4\rightarrow1} = -nC_v(T_a-T_b) = -\Delta U_{4\rightarrow1}
\end{equation}

Note que na \cref{trab_2pra3} o trabalho está sendo feito pelo gás, representando direta e exclusivamente a 
diminuição de sua energia interna. Mas então, W não deveria ser positivo? Ele é! 
Perceba que $(T_b-T_a)<0$ e, portanto, o termo inteiro é positivo!

Note também que o trabalho e variação de energia interna para os dois processos adiabáticos são um o inverso 
do outro. Essa relação será importante na hora de calcular a eficiência, já que, por isso, não precisaremos representá-los.

Agora que definimos os parâmetros mais importantes das quatro etapas, voltaremos à \cref{eficiencia}, e 
substituiremos o $W_{\text{total}}$ pela soma do trabalho feito pelo gás ao longo dos processos isotérmicos 
(lembrando que os processos adiabáticos se anulam), e representaremos $Q_1$ por seu valor dado na \cref{trab_1pra2}:

\begin{equation}\label{eficiencia_expandida}
\text{Eficiência} = \frac{nRT_{a}\ln\left({\frac{V_2}{V_1}}\right) + nRT_{b}\ln\left({\frac{V_4}{V_3}}\right)}{nRT_{a}\ln\left({\frac{V_2}{V_1}}\right)}
\end{equation}

Considerando que os processos $2\rightarrow3$ e $4\rightarrow1$ foram de expansão 
adiabática, podemos escrever:

\begin{equation}\label{adiab_tv_const}
T_2 V_2^{\gamma-1} = T_3 V_3^{\gamma-1},\; \text{e}
\end{equation}

\begin{equation}\label{adiab_tv_const_2}
T_4 V_4^{\gamma-1} = T_1 V_1^{\gamma-1},\; \text{onde}
\end{equation}

\begin{equation}\label{def_gamma}
\gamma = \frac{C_p}{C_v}
\end{equation}

A fim de simplificar Cp, consideremos agora uma transformação isobárica. Então, 
Da primeira lei da termodinâmica, e da definição de trabalho, podemos escrever:

\begin{equation}\label{1_lei_pressao_const}
Q_p=\Delta U_p +W_p=\Delta U_p+P\Delta V
\end{equation}

Onde conhecemos sobre as relações dos gases ideais, e por motivos já discutidos, que:

\begin{equation}\label{lei_gas_ideal_2}
PV=nRT,
\end{equation}

\begin{equation}\label{U_T_ideal}
\Delta U_p = \Delta U_v = nC_v\Delta T    
\end{equation}

Substituindo a \cref{U_T_ideal} e \cref{lei_gas_ideal_2} na \cref{1_lei_pressao_const}, obtemos:

\begin{equation}\label{1_lei_pc_subs}
Q_p = nC_v\Delta T + nR\Delta T = n(C_v+R)\Delta T
\end{equation}

\vspace{0.5cm}

E sabemos que $Q_p = nC_p\Delta T$ (por derivação similar à que fizemos para $C_v$), portanto, 
podemos afirmar que $C_p = C_v + R$\,.

Finalmente, com esse resultado, podemos escrever, da \cref{adiab_tv_const}, da \cref{adiab_tv_const_2} e \cref{def_gamma}:

\begin{equation}\label{adiab_tv_const_3}
\left(\frac{T_2}{T_3}\right)^{C_v/R} = \frac{V_3}{V_2}
\end{equation}

\begin{equation}\label{adiab_tv_const_4}
\left(\frac{T_1}{T_4}\right)^{C_v/R} = \frac{V_4}{V_1}
\end{equation}

A partir da natureza do Ciclo, adicionamos que $T_1 = T_2 = T_a$ e $T_3 = T_4 = T_b$. Igualando então as equações e 
substituindo os termos, concluímos que:

\begin{equation}\label{rel_volume}
\frac{V_3}{V_4} = \frac{V_2}{V_1}
\end{equation}

Substituindo a \cref{rel_volume} na \cref{eficiencia_expandida} obtemos:

\begin{equation}\label{eficiencia_simplificada}
\text{Eficiência} = \frac{nRT_{a}\ln\left({\frac{V_2}{V_1}}\right) + nRT_{b}\ln\left({\frac{V_2}{V_1}}\right)}{nRT_{a}\ln\left({\frac{V_2}{V_1}}\right)}
\end{equation}

Cancelando os termos semelhantes, chegamos a:

\begin{equation}\label{eficiencia_simplificada_2}
\text{Eficiência} = \frac{T_a - T_b}{T_a}
\end{equation}

Ou ainda:

\begin{equation}\label{eficiencia_final}
\eta = 1 - \frac{T_b}{T_a}
\end{equation}

A partir da \cref{eficiencia_final}, observamos então que a eficiência só será 1 (100\%), 
se ${T_b}/{T_a} = 0$. Ou seja, se $T_b = 0$ ou $T_a = \infty$. Ou seja, se a temperatura do reservatório mais frio 
for zero absoluto ou a temperatura do reservatório mais quente for infinita. 

\section{Diagrama P-V}

Nessa seção, derivaremos as leis de formação das funções P(V) a partir da lei dos gases ideais.
Apresentaremos um plot feito em python de um ciclo termodinâmico que segue a mesma ordem
de processos do ciclo de carnot, mas sem a eficiencia perfeita. Já que reservatórios de temperatura
infinita ou zero absoluto não seriam representáveis na figura. 

Usaremos então da geometria e do cálculo para calcular a área embaixo das curvas, podendo-se
assim, alternativamente chegar-se à uma representação algébrica do trabalho realizado no sistema.

\vspace{0.7cm}

Vale ressaltar que para todos os processos utilizaremos o numero de mols $n = 1$, importaremos
a constante R do scipy, usaremos volumes iniciais arbitrários $V_1 = 1L$ e $V_2 = 2L$ e temperatura
dos reservatórios $T_b = 300^\circ K \approx 25^\circ C$ e $ T_a = 600^\circ K \approx 325^\circ C$.

Para $\gamma$, usaremos 5/3, que é o equivalente a usar $C_v = 3/2 \, R$, a capacidade
térmica molar a volume constante para gases monoatômicos.

\subsection{Leis de Formação}

A Lei dos gases ideais já foi extensivamente referenciada nesse documento, mas pelo bem da
compreensão, revisitemos-na:

\begin{equation}\label{lei_gases_ideais_3}
  PV = nRT
\end{equation}

Diferentemente dos processos adiabáticos, é bem simples retirar uma lei de formação dessa equação
para os processos isotérmicos. Considerando que nesses processos, a temperatura deve permanecer constante,
o lado direito inteiro pode ser considerado uma constante. 

Dividindo V dos dois lados, obtemos a primeira lei de formação:

\begin{equation}\label{lei_form_iso}
  P_{iso}(V) = \frac{nRT}{V}
\end{equation}

Para a segunda função, teremos de relembrar uma propriedade das transformações adiabáticas de
gases ideais que é a de que $TV^{\gamma - 1}$ deve ser constante. Se isolarmos T na \cref{lei_gases_ideais_3}
da lei dos gases ideais, e substituirmos esse valor na propriedade mencionada, obtemos:

\begin{equation}\label{gas_ideal_T_Isolado}
  PV = nRT \rightarrow T = \frac{PV}{nR}, \\
\end{equation}

\begin{equation}\label{subst_TV}
  TV^{\gamma - 1} = \frac{PV}{nR} V^{\gamma - 1} = \frac{PV^{\gamma}}{nR}
\end{equation}

Portanto, podemos dizer que $PV^{\gamma - 1}$ também é uma constante, e, por isso:

\vspace{-0.5\baselineskip}

\begin{equation}\label{lei_form_adiab}
  PV^{\gamma} = P_0V_0^{\gamma} \rightarrow P(V) = P_0 \left( \frac{V_0}{V} \right)^{\gamma}
\end{equation}

Da mesma propriedade, definimos os volumes $V_3$ e $V_4$, que serão essenciais para o plot:

\vspace{-0.5\baselineskip}

\begin{equation}\label{vol_3}
  T_b V_3^{\gamma - 1} = T_a V_2^{\gamma - 1} \rightarrow V_3 = V_2 \left(\frac{T_a}{T_b} \right)^{\frac{1}{\gamma -1}},
\end{equation}

\begin{equation}\label{vol_4}
    T_b V_4^{\gamma - 1} = T_a V_1^{\gamma - 1} \rightarrow V_4 = V_1 \left(\frac{T_a}{T_b} \right)^{\frac{1}{\gamma -1}}
\end{equation}

\subsection{Plot do Diagrama}

Com as duas leis de formação generalizadas para os processos isotérmicos
e adiabáticos, podemos plotar cada curva:

\begin{equation}
  P_{1 \rightarrow 2} = \frac{nRT_a}{V_{1 \rightarrow 2}}
\end{equation}

\begin{equation}
  P_{2 \rightarrow 3} = P_2 \left( \frac{V_2}{V_{2 \rightarrow 3}} \right)^{\gamma}
\end{equation}

\begin{equation}
  P_{3 \rightarrow 4} = \frac{nRT_a}{V_{3 \rightarrow 4}}
\end{equation}

\begin{equation}
  P_{4 \rightarrow 1} = P_4 \left( \frac{V_4}{V_{4 \rightarrow 1}} \right)^{\gamma}
\end{equation}

Onde, dada a definição de trabalho na \cref{trab_iso}, perceba que a integral
dessas funções nos dará o trabalho realizado em cada processo. E repare que,
nos processos de compressão, $V_3 > V_4$ e $V_4 > V_1$, logo, $dV < 0$ e $W < 0$.
Nesses casos, o trabalho está sendo feito no gás.

\begin{figure}[H]
   \includegraphics[width=1\linewidth]{diagrama_pv/diagrama_pv}
   \caption{Diagrama P-V do ciclo de Carnot. Note que a área em roxo
   representa tanto o trabalho realizado pelo, quanto o trabalhado realizado no gás. 
   Não é difícil perceber que o gás realiza mais trabalho do que se é realizado nele. Mas, para 
   entender realmente, não só essa diferença, como, novamente, entender a eficiência do sistema,
   calculemos a área da figura fechada.}
   \label{pv_diagram}
\end{figure}

\setlength{\parindent}{1cm}
\setlength{\parskip}{2pt}

\subsection{Derivando a eficiência a partir do diagrama}

Calcularemos o trabalho total realizado no sistema somando as integrais das funções P(V) 
para cada um dos procesoss do ciclo. No fim, dividiremos o resultado pelo calor absorvido que,
como já discutido, sabemos que é igual ao trabalho realizado durante a expansão isotérmica.

\begin{equation}\label{int_sum}
\begin{split}
  W_{total} = \;
  & \left( \int_{V_1}^{V_2} \frac{nRT_a}{V} \, dV + \int_{V_2}^{V_3} P_2 \left( \frac{V_2}{V} \right)^{\gamma} dV \right)\\ + \;
  & \left( \int_{V_3}^{V_4} \frac{nRT_b}{V} \, dV + \int_{V_4}^{V_1} P_4 \left( \frac{V_4}{V} \right)^{\gamma} dV \right)\\
\end{split}
\end{equation}

\vspace{0.5cm}

Onde $V_3>V_4$ e $V_4>V_1$. Logo, podemos substituir as duas últimas integrais pelos seus inversos
com os limites trocados. Além disso, podemos tirar as constantes da integração e ficar com:

\begin{equation}\label{int_sum_2}
\begin{split}
  W_{total} = \;
  & nRT_a \int_{V_1}^{V_2} \frac{1}{V}\,dV + 
  P_2V_2^{\gamma} \int_{V_2}^{V_3} V^{-\gamma} dV \\ - \;
  & nRT_b \int_{V_4}^{V_3} \frac{1}{V}\,dV - 
  P_4V_4^{\gamma} \int_{V_1}^{V_4} V^{-\gamma} dV
\end{split}
\end{equation}

Para melhor compreensão, separemos os dois processos. Para os processos isotérmicos, teremos:

\begin{equation}\label{int_iso}
W_{iso.} = nRT_a \ln{\left(\frac{V_2}{V_1}\right)} - nRT_b \ln{\left(\frac{V_3}{V_4}\right)}
\end{equation}

E para os processos adiabáticos:

\begin{equation}\label{trab_adiab}
    W_{adi.} = \;
    P_2V_2^{\gamma} \int_{V_2}^{V_3} \frac{V^{1 - \gamma}}{1 - \gamma} dV - 
    P_4V_4^{\gamma} \int_{V_1}^{V_4} \frac{V^{1 - \gamma}}{1 - \gamma} dV
\end{equation}
\begin{equation}\label{trab_adiab_2}
  W_{adi.} = \;
  \frac{P_2V_2^{\gamma}}{1 - \gamma} \int_{V_2}^{V_3} V^{1 - \gamma} dV \, - \,
  \frac{P_4V_4^{\gamma}}{1 - \gamma} \int_{V_1}^{V_4} V^{1 - \gamma} dV,
\end{equation}
\begin{equation}\label{trab_adiab_3}
  W_{adi.} = \;
  \frac{P_2V_2^{\gamma}}{1 - \gamma} \, \left[ V_3^{1 - \gamma} - V_2^{1 - \gamma} \right] \, - \,
  \frac{P_4V_4^{\gamma}}{1 - \gamma} \, \left[ V_4^{1 - \gamma} - V_1^{1 - \gamma} \right]
\end{equation}

Botemos então $V_2^{1-\gamma}$ e $V_4^{1 - \gamma}$ em evidência:

\begin{equation}\label{trab_adiab_4}
\begin{split}
  W_{adi.} = \;
       & \frac{P_2V_2^{\gamma}}{1 - \gamma} \, V_2^{1 - \gamma} \left[\left(\frac{V_3}{V_2}\right)^{1 - \gamma} - 1 \right] \\ 
  - \; & \frac{P_4V_4^{\gamma}}{1 - \gamma} \, V_4^{1 - \gamma} \left[1 - \left(\frac{V_1}{V_4}\right)^{1 - \gamma} \right]
\end{split}
\end{equation}

Simplificando e multiplicando em cima e embaixo por -1:

\begin{equation}\label{trab_adiab_5}
  \begin{split}
    W_{adi.} = \;
      &\frac{P_2V_2}{\gamma - 1} \left[1 - \left(\frac{V_3}{V_2}\right)^{1 - \gamma}\right] \\ - \;
    - &\frac{P_4V_4}{\gamma - 1} \left[\left(\frac{V_1}{V_4}\right)^{1 - \gamma}- 1 \right]
  \end{split}
\end{equation}

\vspace{0.3cm}

Realizando a multiplicação distribuitiva:

\begin{equation}\label{trab_adiab_6}
  \begin{split}
    W_{adi.} = \;
    & \frac{P_2V_2 - P_2V_2 \left(\frac{V3}{V2}\right) \left(\frac{V2}{V3}\right)^{\gamma}}{\gamma - 1} \\ - \;
    & \frac{P_4V_4 \left(\frac{V1}{V4}\right) \left(\frac{V4}{V1}\right)^{\gamma} - P_4V_4}{\gamma - 1}
  \end{split}
\end{equation}

Onde sabemos que $PV^{\gamma} = \text{constante}$ e, portanto, $P_2V_2^{\gamma}=P_3V_3^{\gamma}$ e $P_4V_4^{\gamma}=P_1V_1^{\gamma}$. Isolando $P_3$ e $P_1,$
ficamos com:

\begin{equation}\label{trab_adiab_7}
  \begin{split}
  & P_3 = P_2\left(\frac{V_2}{V_3}\right)^{\gamma}, \\
  & P_1 = P_4\left(\frac{V_4}{V_1}\right)^{\gamma}
  \end{split}
\end{equation}

Substituindo na \cref{trab_adiab_6}:

\begin{equation}\label{trab_adiab_8}
W_{adi.} = 
\frac{P_2V_2 - P_3V_3}{\gamma - 1} \; - \;
\frac{P_1V_1 - P_4V_4}{\gamma - 1}
\end{equation}

E finalmente, a partir da \cref{lei_gases_ideais_3}, da lei dos gases ideais:

\begin{equation}
W_{adi.} = 
\frac{nRT_2 - nrT_3}{\gamma - 1} \; - \;
\frac{nRT_1 - nRT_4}{\gamma - 1}
\end{equation}

Substituimos R por seu valor $R = C_v (\gamma - 1)$ e substituimos as temperaturas pelos seus respectivos valores
dada a etapa do ciclo. $(T_1 = T_2 = T_a)$ e $(T_3 = T_4 = T_b)$

\begin{equation}
W_{adi.} = 
\frac{nC_v (\gamma - 1) (T_a - T_b)}{\gamma - 1} \; - \;
\frac{nC_v (\gamma - 1) (T_a - T_b)}{\gamma - 1}
\end{equation}

Simplificando,

\vspace{-0.5\baselineskip}

\begin{equation}
  W_{adi.} = 
  nC_v (T_a - T_b) \; - \; 
  nC_v (T_a - T_b) \; = \;
  0
\end{equation}

Sabendo agora que a soma do trabalho feito nos processos adiabáticos é zero, voltemos a soma dos processos isotérmicos.

\begin{equation}
  W_{total} = W_{iso.} =
  nRT_a \ln{\left(\frac{V_2}{V_1}\right)} - nRT_b \ln{\left(\frac{V_3}{V_4}\right)}
\end{equation}

Como demonstramos no pdf, $V_2/V_1 = V_3/V_4$. Então:

\begin{equation}
  W_{total} = nRT_a \ln{\left(\frac{V_2}{V_1}\right)} - nRT_b \ln{\left(\frac{V_2}{V_1}\right)}
\end{equation}

Para encontrar a eficiência, dividimos esse trabalho total pelo calor absorvido pelo motor que, como discutimos
anteriormente, é igual ao trabalho realizado pela expansão isotérmica.

\vspace{-1\baselineskip}

\begin{equation}
  \frac{W_{total}}{Q_{absorvido}} = \frac{nRT_a \ln{\left(\frac{V_2}{V_1}\right)} - 
  nRT_b \ln{\left(\frac{V_2}{V_1}\right)}}{nRT_a \ln{\left(\frac{V_2}{V_1}\right)}}
\end{equation}

Simplificando o termo semelhante $nR \ln{\left(V_2/V_1\right)}$:

\begin{equation}
  \text{Eficiência} = \frac{T_a - T_b}{T_a} = 1 - \frac{T_b}{T_a}
\end{equation}

\section{Porque um Motor não pode ter 100\% de eficiência}

Um motor nunca pode chegar a 100\% de eficiência pois para isso seria necessário contraditar 2 leis da termodinâmica.

\begin{enumerate}
\item O universo tende ao caos e à entropia, ou seja, a energia sempre quer se dispersar. E quando ela se 
dispersa não pode mais ser usada para gerar trabalho.
\item Energia tem que vir de algum lugar e deve ir a algum lugar, ou seja, não é possível criar energia do 
nada ou mantê-la no mesmo sistema para sempre.
\end{enumerate}

Para que um motor tivesse 100\% de eficiência, seria necessário que a fonte de calor tivesse um valor 
infinitamente alto ou que a fonte fria tivesse o valor igual a 0 
(e um lugar infinitamente grande para o gás expandir). Isso pode ser observado na \cref{eficiencia_final}.

Não existe uma fonte de calor infinita, pois seria necessário algo que vai contra a lei de que a energia tem 
que vir de algum lugar, já que não há energia infinita no universo.

E não existe uma forma de atingir 0 graus Kelvin até onde nós sabemos. Nós conseguimos chegar bem próximo 
em um sistema fechado, mas quanto mais próximo mais difícil. 
Também é impossível criar um recipiente infinitamente longo para que o gás se expanda até 0 Kelvin.

Terceiramente: As condições para chegar a 100\% de eficiência também ignoram muitas leis do nosso mundo. 
Como por exemplo a troca de calor entre o gás e o recipiente, o atrito entre as moléculas ou irradiação 
através do recipiente, e o uso de um gás ideal. Portanto, um motor 100\% eficiente no mundo em que 
vivemos não passa de um sonho.

\section{Como seria um motor 100\% eficiente?}

Um motor 100\% eficiente precisaria de um sistema que não interage com o mundo fora dele em nenhuma forma.

Além disso, precisariamos escolher um entre dois caminhos:

\begin{enumerate}
\item Fonte de calor infinita: Se tivermos algo que gera calor infinito, a quantidade de calor que 
escapa do sistema seria irrelevante pois ele é infinito e sua eficiência seria de 100\%.
\item Fonte de frio capaz de fazer o sistema chegar a 0 Kelvin 
(precisamos também de um recipiente infinitamente grande): 
Se formos capazes de chegar a 0 Kelvin, não perderíamos energia para realizar nenhuma parte do ciclo, 
pois a pressão comprimiria o gás sozinha. Portanto, Eficiência $=100\%$.
\end{enumerate}

Em uma realidade distante da nossa, onde um desses caminhos é possível, 
viável, e verdadeiramente proveitoso, o sistema solucionaria muitos 
problemas enormes de nossas vidas. Vivemos em uma era 
onde energia elétrica é uma necessidade básica e é 
demandada cada vez mais. Um motor 100\% eficiente nos permitiria gerar 
muito mais energia pra suprir nossa demanda, aumentar a acessibilidade 
e reduzir o custo da mesma.

Não só isso, ele diminuiria o aquecimento global em muito já que 
queimaríamos menos combustíveis fosseis, em carros, por exemplo.
Esses, inclusive, não superaqueceriam, pois não há calor vazando, 
mantendo a temperatura do ciclo constante.

\section{Conclusões}

Por fim, entendemos que o ciclo de Carnot nos mostra o porque é impossivel construir 
uma maquina com 100\% de eficiencia. Um modelo pensado à base de uma ideia, ignorando 
conceitos basicos da fisica, mas que, com isso, nos exemplifica a razao dessa 
impossibilidade. Com esse modelo, é possivel avaliar o desempenho de máquinas e 
compreender outros ciclos termodinamicos, além de incentivar a pesquisa na area.

\begin{thebibliography}{9}
\bibitem{1} \href{http://hyperphysics.phy-astr.gsu.edu/hbase/thermo/carnotcon.html#c1}{Nave, C.R. \emph{Carnot Cycle Concepts}. Georgia State University}
\bibitem{2} \href{https://www.scielo.br/j/qn/a/pH47vDT9ydyzwtbPXLwyLhq/?lang=pt&format=pdf}{Nascimento, C.K., Braga, J.P., \& Fabris, J.D. (2004). \emph{Reflexões sobre a contribuição de Carnot à primeira lei da termodinâmica}. SciELO Brasil.}
\bibitem{3} \href{https://www.grc.nasa.gov/www/k-12/airplane/carnot.html}{Hall, N. (2021). \emph{Ideal Carnot Cycle}. NASA Glenn Research Center.}
\bibitem{4} \href{https://www.grc.nasa.gov/www/k-12/airplane/specheat.html}{Hall, N. (2021). \emph{Specific Heats}. NASA Glenn Research Center.}
\bibitem{5} \href{https://www.grc.nasa.gov/www/BGH/realspec.html}{Hall, N. (2021). \emph{Specific Heat Capacity}. NASA Glenn Research Center.}
\bibitem{6} \href{https://chem.libretexts.org/Bookshelves/Physical_and_Theoretical_Chemistry_Textbook_Maps/Supplemental_Modules_(Physical_and_Theoretical_Chemistry)/Thermodynamics/Thermodynamic_Cycles/Carnot_Cycle}{LibreTexts. \emph{Carnot Cycle}.}
\end{thebibliography}

\end{document}